\chapter{Disorder in Quantum Well Structures}

\section{$\mu$PL Results and Analysis for Multiple Quantum Well Samples}
\subsection{Disorder in a Ten Quantum Well Structure}
\indent As a proof-of-concept test for our $\mu$PL experiment, we decided to measure the structural disorder in a ten-period 10nm GaAs QW with 10nm AlGaAs barriers in between wells. We expected to see a large PL signal from this sample. Therefore, it was our goal to measure a PL image of the 10QW sample, with our laser exciting near-resonantly. Our excitation wavelength was 773nm, while our detection center wavelength was 818nm, and the PL signal peaked around 808nm. Figure \ref{raw10qw} shows a single raw image corresponding to a vertical slice of the total PL image. 

\indent From the raw spectrometer images, we found the wavelength corresponding to the maximum PL amplitude and took an image slice from that location on the CCD picture. Figure \ref{slice} shows a representative vertical image slice from the images taken on the 10QW sample. We then took each of these slices, taken from each CCD image corresponding to a position on the total $\mu$PL image, and stacked these slices together according to lens position. From this, we found a total PL image from the 10QW sample, seen in Figure \ref{total10QW}. The features in the PL image correspond to artifacts on the surface of the QW sample: despite our best efforts, surface imperfections and dust were present on the sample during data runs. However, the fact that we are actually able to produce a PL image reconstruction which shows the imperfections as defined reductions in PL signal indicates that the sample surface was in focus, and we are near the maximum image definition we can expect for the disorder map. 

\indent From this image, we can estimate the resolution limit for the PL image alone. Doing so, we calculate that the resolution limit for our imaging system is roughly \ref{}. This is not our goal of 185nm: because the CCD pixels are $13.6\mu$m on each side, our resolution is limited to 612nm by the CCD alone. One can ameliorate this issue by inserting a telescope with two times magnification between the NPBS and the achromatic doublet. An unfortunate side effect of increased resolution, however, is decreased signal strength. It is therefore necessary to increase CCD integration time or increase PL emission by increasing excitation power, but as we were exciting the sample with an already relatively high power of 1mW, increasing the resolution necessitated increasing integration time. 

\indent Though our resolution was less than optimal, we were still able to calculate an energy deviation map for the 10QW sample. To do this, we must find the maximum emission energy for each pixel along our vertical slice, and then calculate the average PL energy for the entire PL image. Then, we find the local energy deviation from the average PL energy by subtracting the average PL energy from the local PL energy. Following this, we obtain our $\mu$PL map. However, this map is too noisy to see structure, as CCD intensity fluctuations of roughly 10 counts or 1\% of the total signal will affect which PL energy is seen as the peak energy at a given location. Therefore, we must take our $\mu$PL energy deviation map and Gaussian smooth the energy deviation values. Doing so, we obtain the energy deviation map seen in Figure \ref{devmap10QW}.


\indent We expected energy deviations as a result of QW width fluctuations to behave according to the following equaiton:
\begin{equation}
\delta E = \frac{h^2 \pi^2}{2 \mu_{HH}}\Big ( \frac{1}{L* \pm a}- \frac{1}{L*^2} \Big )
\end{equation}
where $\delta E$ is the PL energy deviation due to a well width fluctuation of $\pm a$, $\mu_{HH}$ is the heavy-hole reduced mass, and $L*$ is the effective well width accounting for electron wave-function penetration into the barrier CITE Glinka. 

\begin{itemize}
\item 10QW experiments\\
\* PL image and raw img, PL deviation small. Estimate smoothing, estimate resolution, comment on disorder agreement with thry.
\item 4QW experiment\\
\* PL image and raw img, PL deviation slightly larger, but still small estimate, estimate rez, comment on disorder agreement with thry.
\item IQD experiment\\
\* Introduce IQD, talk about disorder in IQD, show raw image, explain difference btwn IQD, MQW, estimate rez, comment on agreement with thry.


\end{itemize}
\section{PLE Results and Analysis}
\begin{itemize}
\item PLE curve analysis\\
show stock PLE curve, show slices, say what they mean, show peaks, say what peaks mean
\item PLE anti-stokes.
show anti-stokes t dep. curves\\
show anti-stokes pow-dep curves, talk about linearity or no, comment on CCD bs., \\
compare with thry, talk about mechanisms. well width dep
\item PLE stokes peak\\
show stokes pow-dep curves, talk about linearity or no, comment on CCD bs., \\
compare with thry, talk about well width dep, say mechanisms understood(?) point to lit chris and I found

\end{itemize}
