\chapter{Theory and Background}
In this chapter, I will provide the theoretical background necessary to understand the work presented in this thesis. The first section will deal with the definition and importance of the bandgap of a material, the second will introduce the effects of confinement on electrons within semiconductors, leading to a discussion of charge carrier confinement within semiconductor quantum wells. The third section will introduce the concept of semiconductor quantum well disorder, and the fourth will motivate the study of semiconductor quantum well disorder with micro photoluminescence spectroscopy. Finally, I will discuss the optical properties of asymetric double quantum wells and how to use photoluminescence excitation spectroscopy to investigate incoherent coupling between excitons in asymmetric double quantum wells.

\section{The Semiconductor Quantum Well}
\subsection{Bandgap}

%bandgap def, a little discussion
\indent In order to understand how electrons behave in a crystalline solid, one must first understand how bound electrons act when arbitrarily many atoms are brought together in a lattice structure. Consider a collection of $N$ atoms sufficiently far apart such that interactions between atoms can be neglected. In this limit, electrons in an atom occupy discreet energy levels. For instance: suppose all $N$ of our atoms are monatomic hydrogen atoms. In our system, an electron behaves according to the following Hamiltonian:
\begin{equation}
\hat{H} = -\frac{\hbar^2}{2m} \bigtriangledown^2 - \frac{e^2}{2\pi \epsilon_0 r}.
\end{equation}
We can find the electron wavefunctions $\psi(x)$ by solving the time independent Shr\"{o}dinger equation,
\begin{equation}
\hat{H}\psi(x) = E_n\psi(x)
\end{equation}
where $E_n$ is the energy of the $n^{th}$ energy level. This equation can be solved using the usual methods \cite{griffiths}, but doing so here would be a diversion, so here are the crucial points: evidently, the $n=1$ ground state electron wavefunction (again neglecting interactions between atoms and ground state perturbations) is
\begin{equation}
\psi(r) = \frac{1}{\sqrt{\pi a_{0}^{3}}}exp\big (-\frac{r}{a_{0}}\big)
\end{equation}
where $a_0$ is the Bohr radius. Note: each $n = 2$ energy level in this system is $N$-fold degenerate, as there are $N$ orbitals with the same energy. Considering just the $n = 1$ states and neglecting perturbations, electrons in these states evidently all have energy
\begin{equation}
E_1 = -13.6eV.
\end{equation}
In the limit that many atoms are brought together such that interactions can no longer be neglected, two things happen: each $N$-fold degenerate electron energy level will split into $N$ components, and these levels will become so close that they will smear into allowed and disallowed energy densities of state \cite{iadonisi, sirdesh, griffiths, davies, fox}. Roughly speaking, the occupied states are known as the ``valance band'' states, and the unoccupied states are known as the ``conduction band states''. The energy difference between the valance band and conduction band is called the bandgap, and its importance will be illuminated momentarily. 

\indent These band states are simultaneous eigenstates of both the Hamiltonian and the crystal momentum \cite{davies, fox}. This means that bands can be easily depicted in momentum space using dispersion curves. Dispersion curves map out allowed band energies in momentum space, and their full functional shape is dependent on the types and arrangement of constituent atoms.  Solids can be broadly organized into three categories based upon the momentum space (or k-space) arrangement of their electron bands and the location of their chemical potential. Crucially, in semiconductors, the bandgap is roughly 1eV and can therefore be optically accessed \cite{fox}. 

\indent Sometimes, as in the case of Gallium Arsenide (GaAS) structures, a local valance band minima and conduction band maxima occur for the same value of $k$ in k-space. Semiconductors with this sort of bandstructure are known as direct-gap semiconductors. Around this value of k, electrons can absorb a photon with enough energy to undergo a direct transition from the valance band to the conduction band \cite{iadonisi, galanthesis}. This fact forms the basis of linear and nonlinear optical studies of semiconductor nanostructures \cite{stevereview} and the operating principle of various photonic devices.

\indent A slight complication to this picture is that there are two valance band curves around this value of k in GaAs, as seen at the gamma point in Figure \ref{GaAsBstruct}. The holes which occupy the valance band energies of less curvature are known as ``heavy holes'' while the holes which occupy the valance band energies of greater curvature are known as ``light holes''. ``Heavy'' and ``light'' refer to the relative effective masses of particles occupying these states. It is the heavy holes that dominate the long-timescale physics we are interested in, so we can ignore the effect of light holes here. 

\begin{figure}[h]
\centering
\includegraphics[width = .4\textwidth]{dispcurve.eps}
\caption{ \doublespacing A typical dispersion curve minima for a direct-gap semiconductor. An optical transition is illustrated at an arbitrary value of $k$, where an electron is absorbing a photon resulting in a transition from the conduction band to the valance band.}
\label{ExampleBands}
\end{figure}

\begin{figure}[h!]
\centering
\includegraphics[width = .3\textwidth]{GaAsBstruct.eps}
\caption{\doublespacing The band structure of GaAs, the allowed states are the thick horizontal curves, and the boxed region is the direct-gap region, in which electrons can be make direct transitions across the bandgap. Note, the extrema of this region look like the dispersion curve in figure 2.1 \cite{davies}.}
\label{GaAsBstruct}
\end{figure}
%dispersion curves


%Discussion of bands in GaAs and AlGaAs, p orbital smearing etc. Find p 63 in davies. (maybe here maybe later)
\newpage
\subsection{Confinement}
%confinement intro
\indent It is well known that nanometer scale confinement of particles results in quantized energy states \cite{griffiths}. In the previous section, I briefly introduced the bandgap, and its important physical properties. In this section, I'll illustrate an interesting application of band theory: the semiconductor quantum well (QW). First, it is important to understand what we mean by confinement, and how quantized energy levels arise for confined particles. I will draw an analogy to a familiar physical situation, the particle confined within an infinite potential. I will then use this analogy to construct a physical picture for QWs, and then I will discuss the formation of excitons and a simple physical model of their behavior, sufficient for understanding the spectroscopy conducted herein.

\indent Perhaps the simplest problem in quantum mechanics is the of confinement of a particle in an infinite, one dimensional potential well. I will sketch a derivation of the wave function of a particle trapped in such a well, and use this derivation as the basis for exploring the physics of the QW exciton. We will begin by considering an arbitrary particle confined in a one dimensional infinite potential well. The potential that our arbitrary particle feels is:

\[ V(x) = \begin{cases} 
      0 & 0 < x < L \\
      \infty &  x<0, ~x > L 
   \end{cases}
\]
where $L$ is the length of the potential well. Graphically, the potential the particle feels looks like Figure 2.3.

\begin{figure}[h!]
\label{infp}
\centering
\includegraphics[width = .6\textwidth]{infpotential.eps}
\caption{\doublespacing A graphical representation of the one-dimensional infinite potential well of width $L$.}
\end{figure}

Our task is to solve the time independent Schr\"{o}dinger equation to show how quantized bound states arise for one-dimensional confinement. The time independent Schr\"{o}dinger equation reads:

\begin{equation}
\hat{H} \psi(x) = E\psi(x)
\end{equation}
where E is the energy of the particle, and $\psi(x)$ is the particle's wavefunction. The particle will evidently be confined to exist in the well, so our Hamiltonian inside the well is just

\begin{equation}
\hat{H} = - \frac{\hbar^2}{2m} \frac{\partial^2}{\partial x^2}
\label{hamil}
\end{equation}
where $m$ is the particle's mass, and $E$ is the particle's  total energy. The time independent Schr\"{o}dinger equation now reads:

\begin{equation} \label{tise}
\frac{\partial^2}{\partial x^2} \psi(x) = - \alpha \psi(x)
\end{equation}
where we define 
\begin{equation}
\alpha = \frac{2mE}{\hbar^2}.
\end{equation}
Because the wave function must be continuous at $x = L$ and $ x = 0$, i.e. it vanishes at those locations, and the potential is odd about the origin,  solutions to eq. 2.7 have the form:
\begin{equation} \label{soln1}
\psi(x) = A sin(k x) 
\end{equation}
where $k$ contains $E$ and is determined by our boundary conditions. Now we want $\psi(L)$ to vanish, but we can't have $A =0$, because that is the trivial solution to eq. 2.7. Therefore, because we want $\psi(L) = Asin(k L) = 0 $, we must have $ka = \pm n \pi$ where $n \in \mathbb{N}$. Now, we can absorb all of the negative combinations of $k L$ into our normalization constant, $A$, and we have, then, that 
\begin{equation}
k_n L = n \pi 
\end{equation}
where the subscript denotes the fact that we now have infinitely many, \textit{discreet} solutions to eq. 2.5. Evidently 
\begin{equation}
k_n = \frac{ n \pi}{L}
\end{equation}
and therefore
\begin{equation}
\psi(x) = A sin(\frac{n \pi x}{L}).
\end{equation}
The boundary condition at $x=L$ sets our value of $k$, and if we plug in eq. 2.9 to eq. 2.7, then we have $k_n^{2} = \alpha$. Now, the allowed energies in the well are 
\begin{equation}
E = \frac{n^2 \pi^2 \hbar^2}{2 m L^2}.
\end{equation}
It will do us no good to normalize the wavefunctions we found, as their use for our purposes is minimal. The important part is that confinement in one dimension resulted in our particle occupying \textit{discreet} energy levels whose energy depends on the physical size of the confinement potential. 

\indent Using layers of semiconductors, one can generate similar one-dimensional confinement effects for electrons. A simple way this can be done is by sandwiching a layer of low-bandgap semiconductor material in between two layers of higher bandgap materials \cite{davies}. One period of this structure is shown in Figure \ref{block}. If the well material is a direct-gap semiconductor, then simple optical transitions (i.e. not mediated by phonons) across the bandgap can be made, as the transition illustrated in Figure 2.1. Figure \ref{GaAsBstruct} is the band structure for GaAs, the chosen well material for the studies of growth disorder. Annotated on the figure is the direct-gap transition zone of interest. The potential well created by the semiconductor sandwich leads to quantization of electron states within the well layer \cite{miller, davies, stevereview}. We can access each of these states optically, making the QW a great testbed for exploring electron dynamics within a simple and well-known potential \cite{stevereview}. 




\begin{figure}[h!]
\centering
\includegraphics[width = .4\textwidth]{Well.eps}
\caption{\doublespacing An example of the semiconductor quantum well. These layers can be repeated arbitrarily many times.}
\label{block}
\end{figure}

\newpage
\subsection{The Exciton}

\indent The simple picture presented above does not quite adequately represent the physics of an electron within a quantum well, however. After an excited electron transitions from the valence band to the conduction band, it will leave behind a vacancy, or ``hole'' \cite{miller, davies}. The electron feels a screened Coulomb potential from this vacancy, as the vacancy is positively charged. I'll briefly sketch why this potential arises and then introduce the key concept of this section: the exciton. Imagine a number of electrons have been excited within the QW. Now, spatially, one will have a quasi-neutral distribution of electrons and holes in the well layer: an electron-hole plasma. I will assume that the holes are stationary relative to the electrons in the well, and that the excitation density is relatively low. These assumptions are \textit{a priori} unphysical, but they immensely simplify the derivation of the electric potential QW electrons feel while preserving the key physics. 

\indent Let's explore the local behavior of an excited electron due to a single adjacent hole in the QW. Note that in this picture, electrons everywhere in the QW feel a Coulomb attraction to the hole, but the electrons adjacent to the hole screen its effects from charges far away. This phenomenon, known in plasma physics as Debye shielding \cite{chen} and in quantum mechanics as electron screening \cite{griffiths}, modifies the pure Coulomb potential one would expect a single electron-hole pair to experience. 

\indent We will assume that the electrons in our electron-hole plasma obey a Maxwellian density distribution, as we are concerned with only long-timescale behavior. Now, the local density of electrons around the hole is 

\begin{equation}
n_e = n_{0}~ exp \Big[ \frac{e\phi}{k T}\Big]
\end{equation}
where $n_{0}$ is the electron density far away, $e$ is the electron charge, $\phi$ is the local electromagnetic potential, and  $T$ is the electron temperature. A complete derivation for this electron number density in an arbitrary plasma can be found in \cite{chen}. This is the local electron distribution, and we can assume $e\phi \ll kT$ because the potential an electron feels due to one hole can be considered small relative to its thermal energy, even at low temperatures. Taylor expanding to first order, we find that
\begin{equation}
n_e \approx n_0 \Big[ \frac{e\phi}{k T}\Big ].
\end{equation}
Now, our local charge density is 
\begin{equation}
\rho(r) = e \Big [ \delta(r) - n_0 \Big( \frac{e\phi}{k T}\Big ) \Big]
\end{equation}
where we've assumed that the hole has positive charge magnitude $e$, is infinitely small, and situated at the origin. In order to figure out what $\phi$ is, we must solve the Poisson equation, assuming quasi static conditions:
\begin{equation}
\bigtriangledown^2 \phi(r)= -\frac{\rho(r)}{\epsilon_{0}}
\end{equation}
 For our charge density, the Poisson equation reads
\begin{equation} \label{pois}
\epsilon_0 \bigtriangledown^2 \phi(r) = -e \Big [ \delta(r) + n_0 \Big( \frac{e\phi}{k T}\Big ) \Big].
\end{equation}
We can define a constant, 
\begin{equation}
k^2 = \frac{n_0 e\phi}{\epsilon_0 k T}
\end{equation}
and now the Poisson equation is
\begin{equation}
\big(\bigtriangledown^2 - k^2 \big) \phi = - \frac{e \delta(r)}{\epsilon_0}.
\end{equation}
This is known as the screened Poisson equation, and its solution is
\begin{equation}
\phi(r) = -\frac{e}{4\pi \epsilon_0 r} e^{-k r}.
\end{equation}
\indent Now, $\phi(r)$, functionally,  the correct result (albeit with a different k) had we proceeded under the more rigorous Thomas-Fermi approximation, assuming only that the potential is weak and variations are slow and smooth over a distance around the hole equivalent to the inverse of the Thomas-Fermi wavevector. This turns out to be a very good approximation for the local potential an electron feels relatively close to a hole in a solid \cite{patterson}. The single-particle hamiltonian for an electron in the QW is now:
\begin{equation}
\hat{H} = - \frac{\hbar^2}{2m} \frac{\partial^2}{\partial x^2} - \frac{e}{4\pi \epsilon_0 r} e^{-k_0 r}
\end{equation}
where $k_0^2 = \frac{n_0 e\phi}{k T_f}$ is the ``corrected'' k for the same potential derived under the Thomas-Fermi approximation \cite{patterson}. 

\indent Note that if the potential electrons feel was exactly Coulombic in nature, then any bound state between an electron and a hole would be hydrogenic \cite{griffiths}. This potential, however, is \textit{not} exactly Culombic in nature, so the bound states between the electrons and holes can't be described by hydrogenic wavefunctions. Nevertheless, bound states between an exited electron and an adjacent hole do exist, and when an electron and hole occupy these states, they form a quasiparticle called an ``exciton", and they can be treated as a single particle with an effective mass \cite{iadonisi, davies}.

\indent The subtleties of the exciton wave function are treated in \cite{iadonisi}, exploring their exact functional form and corresponding density of exciton states in the QW will not be of use to us here. Only energy levels of the QW exciton are discretized by its confinement, those of excitons created in the bulk are not. A final important note: in my derivation, the hole is assumed to be a stationary point particle. This is emphatically not true, but the physics doesn't change that much if we assume both charge carriers are mobile. Excitons can still be treated as a single particle even after removing this assumption. Figure 2.5 depicts a simple physical picture for excitons: the electron and hole excited energy levels can be thought of as the ground state of a particle trapped in a finite potential.  

\begin{figure}[h!]
\label{Spapprox}
\centering
\includegraphics[width = .9\textwidth]{SpApprox.eps}
\caption{\doublespacing A simple model for the behavior of an exciton in a quantum well, suitable for the work completed herein. The black lines correspond to the effective QW potentials for each the electron, hole, and exciton.}
\label{potentials}
\end{figure}


\newpage \section{Quantum Well Disorder}
\indent When an exciton is created, in a direct gap semiconductor such as GaAs, the charge carriers can recombine from the either exciton ``ground'' state ($n=1$), or ``excited''  states ($n>1$) and emit a photon equivalent to the bandgap energy less the exciton binding energy \cite{gilleo}:
\begin{equation}
 E_{emit} = E_{gap}-E_{bind}.
\end{equation}
Since I am interested in linear, long-timescale exciton physics, we can treat QW excitons as an approximately two-level system, only looking at ground state recombination. The emission energies of QW excitons will be dependent on well depth, a function of the barrier composition and well width. The light emitted as a result of carrier recombination after optical excitation is called photoluminescence (PL). Additionally, because the binding energy is small, we cool our samples to low temperature to study exciton PL.

\indent Control of QW layer thickness has improved immensely, as QW structures can be made to precise specifications with modern molecular beam epitaxy \cite{davies}. However, imperfections of layer width on the order of a crystal monolayer occur unavoidably at the interface between the well and barrier materials during the QW manufacturing process \cite{yoshitaterrace,weis, glinka}. These defects, a form of structural disorder, slightly change the width of the well layer and thus subtly modulate exciton emission energies, an illustration of which is shown in Figure 2.6. By analogy to eq. 2.11, excitons localized in slightly thinner than average sections of the QW will emit at slightly higher energies than average. Conversely, excitons localized in thicker than average sections of the QW will emit at slightly lower energies than average. Thus, structural disorder is the main contribution to inhomogeneous broadening in the QW coherent optical response \cite{bristowsep}. 

\begin{figure}[t!]
\label{rel-thickness}
\centering
\includegraphics[width = .9\textwidth]{rel-thickness.eps}
\caption{\doublespacing A graphical representation of the effective QW exciton potentials (black lines) and QW exciton ground states in wells of various thickness. In a), the energy of a ground state exciton located in a portion of the QW of average thickness, the blue line in b) depicts the ground state energy of an exciton located in a slightly thinner than average portion of the QW,  and in c), the blue line depicts the ground state energy of an exciton located in a slightly thicker than average portion of the QW. In both b) and c), the average exciton ground state energy is depicted by the maroon dashed line.}
\end{figure}

\section{Exciton Coupling in Asymmetric Double Quantum Wells}

\indent Quantum coupling between excitons occurs when multiple quantum wells get close enough so that the exciton wavefunction can tunnel slightly into adjacent wells \cite{griffiths, davies}. In order to study quantum coupling between states in adjacent QWs, however, it is not simply enough to grow multiple quantum well layers fairly close to one another. Evidently, if each of the wells is of identical thickness, then QW PL  from one well will be spectrally indistinguishable from another. Indeed, it is necessary to grow wells of varying thickness when studying coupling in multiple quantum wells \cite{hegartycouple}.  

\indent Asymmetric multiple quantum well (AQW) samples are a convenient system in which to study exciton coupling between wells. Incoherent (long timescale) coupling between exciton states can occur through dipole-dipole interactions \cite{tomita}, thermal activation \cite{Borri} or other incoherent processes. When excitons absorb light resonant to the narrow well (NW) exciton ground state and emit light resonant to the wide well (WW) exciton ground state, then the coupling between those two wells happened in the Stokes direction. Anti-Stokes coupling is just the reverse of this process: WW absorption and NW emission. In other words, if we see PL peaks that correspond to absorption in one well and emission from the other well, we can quantify the intensity of that coupling. Understanding incoherent coupling between excitons in AQWs is of interest because an improved understanding of the various processes that govern carrier transfer in nanostructures can help improve the efficiency and quality of nanostructure devices like photodiodes and transistors. Additionally, an increased understanding of exciton coupling processes is of fundamental research interest.



\section{Microphotoluminescence Spectroscopy}
\indent Because local QW thickness determines exciton emission energy, a spatial picture of disorder is possible through spectral imaging. By using a continuous wave (CW) excitation source to create a population of QW excitons and monitoring the emission energy as a function of sample position, one can extract the local QW width \cite{bristowsep}. With sufficiently high resolution, obtaining a map of emission energies for a representative portion of a QW sample is possible. In order to obtain a map of emission energies, one must monitor the photoluminescence (PL) energies as a function of position. In a PL experiment, excitation light is directed upon a sample, exciting a large number of charge carriers to the conduction band. In a PL experiment on a QW, we create a large number of excitons, and when they recombine, they emit a photon. If we regard the exciton as a two level system, by exciting QW excitons near-resonantly, the PL energy will be a function of local well width. The key point is this: by obtaining a spatial PL signal, and then resolving the PL energy at each location, we obtain a measure of local QW width and therefore structural disorder.

\indent In order to obtain an image of PL, we must collect the signal carefully. More precisely, if we are to obtain a spatial picture of QW disorder, we must collect and spectrally resolve a PL image. The PL will be emitted from the exciton population over a $4 \pi$ solid angle, so we must construct an imaging system and place the QW directly at the focus to obtain a clear PL image. Furthermore, since the scale of the disorder is on the order of 100nm \cite{yoshitaterrace}, we must have comparable resolution for the PL image. A system capable of resolving a PL image is shown in Figure 2.7. The magnification of the PL image is set by the ratio of the lens focal lengths. Namely:
\begin{equation}
M = \frac{f_1}{f_2}
\end{equation}
where $f_2$ and $f_1$ are the focal lengths of the short focal length lens and the long focal length lens respectively, and $M$ is the image magnification factor. A PL experiment with sub-micron resolution which is capable of spectrally resolving a PL image is known as a ``microphotoluminescence spectroscopy" experiment, or $\mu$PL.
\begin{figure}[h!]
\label{confocal}
\centering
\includegraphics[width = .3\textwidth]{confocal1.png}
\caption{\doublespacing A representation of a confocal optical geometry used to collect the PL from the QW sample. The PL image is being collected from a small region of the QW sample, magnified, and then collimated by the two lenses.}
\end{figure}

\indent As simple as the PL image collection is, it is difficult to obtain the requisite image resolution because the PL is around 750-800nm wavelength, but as asserted above, the islands can be smaller than this. The Abbe diffraction limit is
\begin{equation}
d = \frac{\lambda}{2nNA}
\end{equation}
where $\lambda$ is the wavelength of the PL image, $n$ is the index of refraction in the intermediate space between the sample and lens 2 (the imaging lens) in Figure 2.7, and $NA$ is the numerical aperture of lens 2. Note that $NA = f / D$ where $f$ is the focal length of the imaging lens and $D$ is its diameter. In our case, $d \approx 500 nm$ for $\lambda \approx 780 nm$, $n = 1$ in vacuum, and $NA = 0.83$, as that was the NA of the lens in our experimental setup. Even if we were able to operate our imaging system at the Abbe diffraction limit, we still wouldn't have the resolution necessary to resolve adjacent disorder sites.

\indent In order to get around this limit, we either need to resort to exotic microscopic techniques, or we can employ a fairly simple trick. It has been shown that by increasing the index of refraction ($n$) with a solid immersion lens (SIL), the Abbe diffraction limit can be substantially reduced  \cite{yoshitaapp} paper, and broad SIL paper. In order to do this, one simply places a hemisphere of sufficiently high $n$ material between the sample and the imaging lens. Figure 2.8 is a diagrammatic representation of this improvement. Note, I'll use SIL and hemisphere interchangeably from here on out, though they aren't necessarily interchangeable, as ``SIL'' refers to type of lens, generally a truncated sphere of some degree. 

\indent In my experimental setup, I used a Zinc Selenide (ZnSe) SIL, for which $n = 2.4$ at $\lambda = 780 nm$. This improvement decreases the diffraction limit from $d \approx 500nm$ to $ d \approx 185 nm$, roughly sufficient resolution for our purposes. I'll expand on the precise setup in the next chapter, but the experiment will be very similar to what I've just described.
\begin{figure}[h!]
\label{confocal2}
\centering
\includegraphics[width = .3\textwidth]{confocal2.png}
\caption{\doublespacing A representation of the principle of our specific $\mu$PL experiment. The PL image leaves the SIL normal to the hemispheric surface, so it just increases $n$ and lowers the diffraction limit.}
\end{figure}


\section{Photoluminescence Excitation Spectroscopy}
\indent  Photoluminescence Excitation spectroscopy (PLE), a linear technique, can provide information on the energies and locations of various exciton states in AQWs. PLE is a simpler method than $\mu$PL, as only the relative PL intensity is monitored. Generally, one monitors PL at a specific detection wavelength as the excitation wavelength is scanned across a wide range. By monitoring the PL as a function of excitation wavelength, we obtain a measure of absorption by the sample \cite{fox}. By extending this technique slightly, we can quantify the coupling between excitons in AQW samples. In order to do this, we scan the excitation wavelength of our light source, and take a PL spectrum (not just a single intensity value) for each different excitation wavelength. That way, the emission from both wells can be monitored as a function of excitation wavelength, and coupling between two wells can be quantified: as the excitation wavelength scans over an exciton resonance in one well, the PL signal from the other well can be monitored to see if that resonant excitation at one energy increases PL at the other. 
