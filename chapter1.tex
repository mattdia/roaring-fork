\chapter{Introduction}

%situating importance
\indent Semiconductor nanostructures provide uniquely manipulable platforms for probing quantum interactions between light and matter in a highly controllable environment. These structures typically require the ability to manipulate semiconductor growth on the nanometer scale. As a consequence, manufacturing such materials is a complex and involved process. However, due to the unique physical properties of valence electrons within the crystalline latice, describing the physics of semiconductor nanostructures is relatively simple [CITE davies]. Therefore, within the broader context of atomic and molecular optics, materials physics, and ultrafast spectroscopy, semiconductor nanostructures form the basis of deeper study into many-body physics in the quantum-mechanical regime.

%defining bandgap and confinement
\indent Band theory, broadly construed, describes how electrons behave as atoms group together to form various materials. As this introduction is not meant to be a rigorous description of condensed matter physics, a brief sketch of 
band theory will be sufficient to explain the importance of semiconductor nanostructures. When a large number of atoms coalesce, the discrete energy states of electrons relative to their host atoms smear into 'bands' of allowed states electrons can occupy in the atomic superstructure. In this limit, two distinct energy bands form: the valence band (lower electron energies) and the conduction band (higher electron energies). Electrons occupying the valence band are confined to the atomic superstructure, whereas electrons within the conduction band states may roam around the superstructure. The difference between these two energies in a material is known as its 'bandgap'. In semiconductors, the bandgap is relatively small, and thus one can photo-excite an electron from the valence band to the conduction band. Electronic confinement happens when one takes a semiconductor material of lower bandgap and surrounds it with a semiconductor of higher bandgap on the length scale of the electronic wavefunction. The confinement dimensionality (generally one, but sometimes two or three), width, and energetic depth are all free parameters in this process. Because of this parametric freedom, everything from simple confinement schemes to complicated structural configurations have proven to be very useful for creating electronic and opto-electronic devices CITE NAT REVIEW. In addition, semiconductor nanostructures form the basis for photonic quantum information devices CITE DEVICE REVIEW, QI REVIEW.

\indent For our purposes, we will be discussing semiconductor quantum wells (QWs). These nanostructures are layers of relatively low bandgap material (GaAs in our case) sandwiched between layers of higher bandgap (AlGaAs). If one excites an electron within the confined layer, that electron will be confined to move about in two dimensions. Furthermore, that electron will be weakly bound to the vacancy (or 'hole') it left in its parent atom. This electron-hole pair constitute a quasiparticle known as an exciton, the importance of which will be expanded upon later. Excitons confined within QWs can be treated theoretically using a simple particle-in-a-box quantum mechanical picture, making QWs useful devices for studying the subtleties of lights' interaction with matter, among other things.

%defining disorder
\indent Despite immense progress in the growth of quantum wells, manufacturing processes still unintentionally introduce inhomogeneities to the layer thickness during crystal construction. This translates to small fluctuations in the width of the interface between two different materials on the scale of a few crystal monolayers. What this means for a quantum well structure is that small fluctuations in interface flatness translate to small fluctuations in well width as these happen at both interfaces of the quantum well and barrier layers. These small width fluctuations are known as disorder.

%motivating study of disorder
\indent Semiconductor quantum wells are studied optically by our lab. In most of our experiments involving these structures, excitons are generated in the wells, and their dynamic behavior can be monitored by several different spectroscopic techniques, notably Multidimensional Coherent Spectroscopy (MDCS). As will be shown later, small changes in well width cause changes in confinement potentials for excitons and thus affect the energies at which excitonic states exist in the wells. Therefore, disorder is the main cause of inhomogeneous broadening within QW samples. This is known to be true CITE BRISTOW, but quantifying the spatial distribution of disorder is important because complete information concerning inhomogeneous contributions to the overall spectral response of QWs can help us enrich our understanding of many-body physics of excitons within quantum wells. In this thesis, I present my development of a method known as micro-photoluminescence spectroscopy to obtain a spatial map of emission energy fluctuations. Furthermore, I translate this spatial picture into a quantitative measurement of disorder on three different types of QW structures: a periodic ten-quantum well structure, a periodic four-quantum well structure, and an interfacial quantum dot ensemble. 