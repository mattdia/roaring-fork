\chapter{Introduction}

%situating importance

\indent Semiconductor nanostructures provide uniquely manipulable platforms for probing quantum interactions between light and matter in a highly controllable environment. These structures typically require the ability to manipulate semiconductor growth on the nanometer scale. As a consequence, manufacturing such materials is a complex and involved process. One might expect that describing electron physics within semiconductors would be an intractable problem. Indeed, exactly solving for the motion of a single electron within a crystal would be impossible, as it would require solving a system of $10^23$ coupled differential equations.  However, the Pauli exclusion principle allows for an easy way out of this problem. Because each electron in a material must occupy a different eigenstate than every other electron within that material, we can treat the behavior of electrons within crystalline lattices as a one particle problem [CITE davies]. 

%defining bandgap and confinement
\indent Band theory, broadly construed, is the solution to this one particle problem. As this introduction is not meant to be a rigorous description of condensed matter physics, a brief sketch of band theory will be sufficient to explain the importance of semiconductor nanostructures. When a large number of atoms coalesce, the discrete electron states of smear into ``bands'' of allowed states in the atomic superstructure. In this limit, two distinct energy bands form: the valence band (lower electron energies) and the conduction band (higher electron energies). The difference between these two energies in a material is known as its ``bandgap". In semiconductors, the bandgap is ten times smaller than that of an insulator [CITE either Griffiths or Fox]. Indeed, the bandgap of many semiconductor materials is in the visible range, and thus one can photo-excite an electron from the valence band to the conduction band.

\indent The ability to optically access a variety of states fact makes semiconductor materials ideal building blocks for optical devices. For instance, when one confines an electron in a semiconductor, the states which that electron can occupy in the conduction band of the confined material become quantized CITE Davies. Therefore, low-dimensional semiconductors make an ideal testbed for linear and nonlinear spectroscopy studies of electron dynamics in solids.

\indent Semiconductor nanostructures are often the preferred platform for these investigations. Nanoscale manipulation of material properties allows for the manufacture of a wide array of semiconductor devices. In such devices, the confinement dimensionality (generally one, but sometimes two or three), width, and energetic depth are all free parameters. Because of this parametric freedom, everything from simple confinement schemes to complicated structural configurations have proven to be very useful for creating electronic and opto-electronic devices CITE NAT REVIEW. In addition, semiconductor nanostructures form the basis for photonic quantum information devices CITE DEVICE REVIEW, QI REVIEW.

\indent For our purposes, we will be discussing semiconductor quantum wells (QWs). These nanostructures are layers of relatively low bandgap material (GaAs in our case) sandwiched between layers of higher bandgap ($AlGaAs_{subscripts relevant for distinguishing between samples, I think}$). If one excites an electron wto the conduction band of a material, the electron can be weakly bound to the vacancy (or ``hole'') it left in its parent atom. This bound electron-hole pair is known as an excition, and its importance will be elaborated on later. If an exciton is created within the QW, it can be treated theoretically using a simple particle-in-a-box quantum mechanical picture. For our purposes, QWs are extremely useful for studying the subtleties of light's interaction with matter.

%defining disorder
\indent Despite immense progress in the growth of quantum wells, manufacturing processes still unintentionally introduce inhomogeneities to the layer thickness during crystal deposition. This translates to an uneven interface between two different materials on the scale of a few crystal monolayers. What this means for a quantum well structure is that small fluctuations in interface flatness translate to small fluctuations in well width as these happen at both interfaces of the quantum well and barrier layers. These small width fluctuations are known as structural disorder.

%motivating study of disorder
\indent Semiconductor quantum wells are studied optically by our lab. In most of our experiments involving these structures, excitons are generated in the wells, and their dynamic behavior can be monitored by several different spectroscopic techniques, notably various linear spectroscopic methods and multidimensional Coherent Spectroscopy (MDCS). As will be shown later, small changes in well width cause changes in confinement potentials for excitons and thus affect the energies at which exciton states exist in the wells. Therefore, disorder is the main cause of inhomogeneous broadening within QW samples. This is known to be true (CITE BRISTOW), but quantifying the spatial distribution of disorder is important because complete information concerning inhomogeneous contributions to the overall spectral response of QWs can help us enrich our understanding of many-body physics of excitons within quantum wells. In this thesis, I present my development of  micro-photoluminescence spectroscopy ($\mu PL$), a type of linear spectroscopy, to obtain a spatial map of emission energy fluctuations. Furthermore, I translate this spatial picture into a quantitative measurement of disorder on three different types of QW structures: a periodic ten-quantum well structure, a periodic four-quantum well structure, and an interfacial quantum dot ensemble. 

%defining the AQW
\indent In addition to studying quantum well disorder, linear spectroscopy is useful for characterizing the energy profile of electron states in matter. For example, it is useful to study exciton states the asymmetric double quantum well, as these systems are great testbeds for various quantum coupling phenomena CITE AQW Review. Understanding incoherent coupling between exciton states in asymmetric double quantum wells allows us to fulfill two important experimental goals: we obtain a better understanding of the exact absorption energies of the various exciton states, and we can explore incoherent coupling between states in each of the wells. In addition to $\mu$-PL experiments, I present the development of Photoluminescence Excitation Spectroscopy (PLE) to study the linear properties of exciton states within AQWs.

\indent The thesis is structured such that Chapter 2 is a theoretical introduction to the physical concepts necessary to understand our $\mu$-PL and PLE experiments. Chapter 3 will be a description of our experimental methods, while results of our experiments will be presented in Chapter 4. Chapter 5 will conclude this thesis with our experimental interpretations and a discussion of possible experimental directions for the future. 







