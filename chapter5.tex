\chapter{Conclusion and Future Work}
\section{Conclusions from Linear Spectroscopic Experiments}
\indent In this thesis I've presented the development of a way to obtain a spatial and spectral picture of semiconductor quantum well disorder. I have shown, furthermore, that we can obtain the desired quantification of disorder for a variety of QW systems. However, due to spectral averaging, this method is unable to distinguish the disorder contributions from individual wells in MQW samples. This was to be expected, however, as over more than two quantum wells, the disorder effects on PL energies smear out and have the effect of broadening PL line-widths.

\indent In more disordered systems, like IQDs, $\mu$PL provides a unique way to measure spatial disorder. When the effects of spectral imaging can be minimized, such as in our IQD samples or in single QW samples, this method can provide adequate spatial resolution to resolve the effects of disorder on PL emission energies at the small disorder size scales. 

\indent Taken as a whole: I showed, with the data taken on the 4QW system, that with a sufficiently high magnification, it is possible to achieve the spatial resolution necessary to image structural disorder. Furthermore, with the data taken on the IQD system, I showed that our method has the ability to provide a spectrally resolved picture of PL emission. Therefore, with the correct samples (highly disordered QWs or presumably single QWs), we can use this method successfully to quantify structural disorder in semiconductor nanostructures.

\indent I showed that incoherent coupling between AQW excitons can be studied in detail with our enhanced PLE measurements. The richness of the data we retrieve using a spectrometer allows us study coupling in both the Stokes and anti-Stokes directions simultaneously. When employed to study the temperature dependance and barrier width dependance of incoherent coupling between QW excitons, we showed that our enhanced PLE technique is capable of producing a wealth of data about coupling strength and PL line-shapes that otherwise would not be possible.

\section{Future Work}
\indent A natural next step for the $\mu$PL experiments we conducted would be to study single quantum well samples to see if we can, as in the case with IQDs, extract adequate spatial and spectral data from the PL signal to quantify the disorder of a single well, as that result is more easily generalizable than the IQD $\mu$PL result. Additionally, it would be useful to combine $\mu$PL data taken, on MQW or other QW samples, with two-dimensional Fourier transform spectroscopy taken on the same samples to see how localization and disorder affect the nonlinear optical response of QW excitons.

\indent Our PLE studies of AQW systems proved useful in locating exciton states and quantifying incoherent coupling between exciton states in the adjacent wells. We could further our understanding of incoherent coupling by developing a better understanding for thermal mediation of coupling that adequately treats both temperature and barrier width dependance, as authors have independently addressed those questions in the past \cite{borri, batsch, tomita}.
