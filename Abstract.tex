\documentclass[letterpaper,14pt,aps,reprint,notitlepage,onecolumn]{revtex4-1}
\usepackage{times}
\usepackage{graphicx}
\usepackage{amsmath}
\usepackage{amssymb}
\usepackage{epstopdf}
\usepackage{natbib}
\usepackage{graphicx}
\usepackage{setspace}
\linespread{2}

\begin{document}

\title{Spatially Resolved Spectroscopy of Semiconductor Nanostructure Disorder}
\author{M. W. Day, R. Sechrist, R. Singh, C. L. Smallwood, S. T. Cundiff}
\vspace{.5cm}
\address{$^1$ JILA, University of Colorado \& National Institute of Standards and Technology, Boulder CO 80309, USA}
\address{$^2$ Department of Physics, University of Michigan, Ann Arbor, MI 48109, USA}
\maketitle
\vspace{0cm}

Manufacturing processes unintentionally introduce inhomogeneities in the width of semiconductor quantum wells on the order of  a few monolayers. These fluctuations, known as disorder, modulate optical transition energies of excitons confined within the quantum well layer. Disorder is the main cause of spectral inhomogeneous broadening, and it is therefore imperative to quantify this disorder so its effect on exciton confinement potentials can be accounted for in ultrafast spectroscopic studies on semiconductor quantum wells. We present the use of micro-photoluminescence spectroscopy to accomplish this goal, as well as the results of our study so far.

\begin{figure}[h!]
\begin{center}
\includegraphics[width=\textwidth]{4QWdE.pdf}
\caption{Micro photoluminescence ($\mu$ PL) spectra taken on periodic 4 AlGaAs/GaAs/AlGaAs quantum well structure.}
\end{center}
\end{figure}



\end{document}
