\chapter{Introduction}
\indent Recent advances in nanoscale semiconductor manufacturing techniques allow for the realization of unique nanostructures ideally suited for a broad range of optical and electronic devices. Precise control of material geometry and immense parametric freedom during device fabrication makes tailoring semiconductor nanostructures to their applications relatively easy. With these processes, confining charge carriers on the scale of their de Broglie wavelengths becomes possible. Nanostructures employing confinement on such small scales have proven to be especially useful. A particularly useful subset of these structures, known as quantum wells, confine their charge carriers in just one dimension. It is therefore possible to create such useful devices as highly efficient photo- and laser diodes. Additionally quantum wells (QWs) have shown particular promise in such diverse areas as quantum information processing  CITE QI REVIEW, as low noise transistors, and as saturable absorbers in mode-locked lasers. In addition to their applications, QWs have proven useful as a testbed for studying the quantum interactions between light and matter CITE Steve Review.

\indent Despite immense progress in the growth of quantum wells, manufacturing processes still unintentionally introduce inhomogeneities to the layer thickness during crystal deposition. This translates to an uneven interface between two different materials on the scale of a few crystal monolayers. What this means for a quantum well structure is that small fluctuations in interface flatness translate to small fluctuations in well width as these happen at both interfaces of the quantum well and barrier layers. These small width fluctuations are known as structural disorder.

\indent Disorder affects charge carrier mobility within QWs, which can affect device efficiency and optical activity. Because of the ubiquity of QW structures, it is important to characterize disorder. Being able to quantitatively map structural disorder within QWs can help manufacturers improve the uniformity of their QWs, and in turn the quality and efficiency of their devices. For use of QWs in optical experiments, quantifying the spatial distribution of disorder is important because complete information concerning inhomogeneous contributions to the overall spectral response of QWs can help us enrich our understanding of many-body physics of excitons within quantum wells.

\indent The local thickness of a QW structure can be optically determined: simply by spectrally resolving the total emission signal from a photo-excited QW, allows one to determine the average thickness of a QW layer CITE Gilleo. Structural QW disorder modifies the thickness on the well at nanometer scale and therefore slightly modulates the energies of the emitted photons as a function of position within the QW system. Therefore, disorder can be experimentally quantified. By obtaining a spatial map of QW emission energies with sufficient resolution spatial and spectral resolution, we can determine the local well thickness and therefore obtain a quantitative picture of disorder within a QW. By optically exciting a QW sample and spectrally imaging the emitted light, a linear spectroscopic technique known as micro-photoluminescence spectroscopy, we recover a spatially resolved map of emission energies, and thereby quantify QW structural disorder CITE Yoshita SIL/QW paper.

\indent In addition to studying quantum well disorder, linear spectroscopy is useful for characterizing the energy profile of electron states in matter. For example, we can use linear spectroscopy to study exciton states and incoherent coupling in the asymmetric double quantum well (AQW), as these systems are great testbeds for various quantum coupling phenomena CITE AQW Review. Understanding incoherent coupling between exciton states in asymmetric double quantum wells allows us to fulfill two important experimental goals: we obtain a better understanding of the exact absorption energies of the various exciton states, and we can explore incoherent coupling between states in each of the wells. We can 

\indent In this thesis, I present the development of a high-resolution micro-photoluminescence spectroscopy ($\mu$-PL) experiment to quantify QW disorder. In addition to $\mu$-PL experiments, I present the development of Photoluminescence Excitation Spectroscopy (PLE) to study the linear properties of exciton states within AQWs.