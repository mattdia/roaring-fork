\chapter{Introduction}



\indent Nanoscale semiconductor manufacturing techniques allow for the realization of unique nanostructures ideally suited for a broad range of optical and electronic devices. Precise control of material geometry and immense parametric freedom during device fabrication makes tailoring semiconductor nanostructures to their applications relatively easy. With these processes, confining charge carriers on the scale of their de Broglie wavelengths becomes possible. Nanostructures employing confinement on such small scales have proven to be especially useful. A subset of these structures, known as quantum wells, confine their charge carriers in just one dimension. It is therefore possible to create devices like highly efficient photo- and laser diodes. Additionally quantum wells (QWs) have shown particular promise as low noise transistors, and as saturable absorbers in mode-locked lasers, among other uses. In addition to their applications, QWs have proven useful as a testbed for studying the quantum interactions between light and matter \cite{stevereview, davies}.

\indent Despite immense progress in the growth of QWs, manufacturing processes still unintentionally introduce inhomogeneities to the layer thickness during crystal deposition. This translates to an uneven interface between two different materials on the scale of a few crystal monolayers. For a quantum well structure, small fluctuations in interface flatness translate to small fluctuations in well width as these happen at both interfaces of the quantum well and barrier layers. These small width fluctuations are a form of structural disorder.

\indent Disorder affects charge carrier mobility within QWs, which can affect device efficiency and optical activity. Because of the ubiquity of QW structures, it is important to characterize disorder. Being able to quantitatively map structural disorder within QWs can help manufacturers improve the uniformity of their QWs, and in turn the quality and efficiency of their devices. For use of QWs in optical experiments, quantifying the spatial distribution of disorder is important because complete information concerning inhomogeneous contributions to the overall spectral response can help us enrich our understanding of many-body exciton interactions.

\indent The local thickness of a QW structure can be optically determined: simply by spectrally resolving the total emission signal from a photo-excited QW, allows one to determine the average thickness of a QW layer \cite{gilleo}. Structural QW disorder modifies the thickness of wells and therefore slightly modulates the energies of quantum well exciton states as a function of position within the well. Therefore, disorder can be experimentally quantified. By obtaining a spatial map of QW emission energies with sufficient resolution spatial and spectral resolution, we can determine the local well thickness and therefore obtain a quantitative picture of disorder within a QW. To do this, we optically excite a QW sample and spectrally imaging the emitted light, a linear spectroscopic technique known as micro-photoluminescence spectroscopy. Doing so, we recover a spatially resolved map of emission energies, and thereby quantify QW structural disorder \cite{yoshitaapp}.

\indent In addition to studying quantum well disorder, linear spectroscopy is useful for characterizing the energy profile of electron states in matter. For example, we can use linear spectroscopy to study exciton states and incoherent coupling in the asymmetric double quantum well (AQW). Due to the geometrically tunable of AQW structures, these systems are great testbeds for various quantum coupling phenomena, and understanding incoherent coupling between exciton states in asymmetric double quantum wells allows us to fulfill two important experimental goals: we obtain a better understanding of the exact absorption energies of the various exciton states, and we can explore incoherent coupling mechanisms between exciton states in each of the wells. In particular, we can explore the temperature and barrier width dependence of coupling between exciton states in AQW samples.


\indent In this thesis, I present the development of a high-resolution micro-photoluminescence spectroscopy ($\mu$-PL) experiment to quantify QW disorder. Furthermore, I employ this experiment to produce a quantitative measurement of disorder on three different types of QW structures: a periodic ten-quantum well structure, a periodic four-quantum well structure, and an interfacial quantum dot ensemble.  In addition to $\mu$-PL experiments, I present the development of photoluminescence excitation spectroscopy (PLE) to study the incoherent coupling between exciton states in Gallium Arsenide/Indium Gallium Arsenide AQW structures. I will then use our PLE characterization of exciton states in AQWs to comment on and add to the relative paucity of information on thermal and barrier width mediation of coupling between exciton states in the stokes and anti-stokes directions. 

\indent This thesis is structured such that Chapter 2 is a theoretical introduction to the physical concepts necessary to understand our $\mu$-PL and PLE experiments. Chapter 3 will be a description of our experimental methods, while results of our experiments will be presented in Chapter 4. Chapter 5 will conclude this thesis with our experimental interpretations and a discussion of possible experimental directions for the future. 