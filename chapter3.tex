\chapter{Experimental Methods}
In this chapter I will describe the experiments I conducted for this thesis. The first section will deal cover the configuration and customization of the continuous wave (CW) Titanium Sapphire laser I used as an excitation source for conducting both PLE and $\mu$PL experiments. Section two will cover the design and construction of the in-cryostat optic mount for the $\mu$PL experiment as well as the optical configuration for data collection. Section three will illustrate the data collection procedure used in $\mu$PL experiments, as well as the function and implementation of LabView code I wrote for hardware control and data acquisition. In section four, I will lay out the optical design of our PLE experiments and in section five, I will discuss the experimental data collection process and signal optimization routines.

\section{The Light Source}
\indent The PLE and $\mu$PL experiments required a CW laser light source with a few properties: the laser must be a stable and fairly high-power light source with narrow line-width. For $\mu$PL, it was important that we have a fairly Gaussian and symmetric beam so we could obtain the desired spot-size and resolution at the sample. Additionally, conducting PLE scans required that we have the ability to computer control the laser wavelength over a fairly broad range of wavelengths, roughly a spectral region from $\lambda = 780$nm to $\lambda = 850$nm. The laser we chose for this task was a Schwartz Electro Optics Titan-CW Titanium Sapphire (Ti:Sapph) laser. Its specifications were fairly close to our needs, as its specified operating power is 500mW with a tunable range from roughly 700-820nm CITE Titan manual. 

\indent Though the Ti:Sapph laser nearly met our specifications, it required two modifications to be operable in our experiments. First, we needed to add a computer controlled actuator to rotate the the birefringent tuner in order to allow for increased tenability and repeatability relative to a manually actuated micrometer. We modified the rotation mount for the birefringent filter (BRF) to accommodate a Newport TRB25 linear actuator. The linear actuator pushed a spring-loaded arm to rotate the BRF to a specified angle and select our desired wavelength. Figure 3.1 depicts the modified BRF mount with Newport actuator